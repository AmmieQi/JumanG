% Juneki Hong and Michael Tango
% Declarative Methods
% Professor Eisner
% March 4, 2013


\documentclass{article}
\author{Juneki Hong and Michael Tango}
\title{Declarative Methods Project Proposal: \\ JumanG}
\usepackage[ampersand]{easylist}

\begin{document}

\maketitle


VERSION 2


\newpage

\subsection*{Introduction}

We will be creating a declarative language for Data Visualization. Specifically for visualizing dynamically changing graphs.
JumanG will be able to derive a graph out of specifications over data, and redraw the graph whenever the specifications change.



\subsection*{Frontend}

According to Tim, a “change” in a graph should in reality be a change to the underlying data that the graph represents. Because of this, we would want JumanG be a language about relationships between data. Each variable declaration would specify a piece of data and its attributes. For example:

\begin{verbatim}
// Alice, Bob, and Casper are data that have these attributes
Alice   [type: human, foo: bar, foods:  [apples, broccoli, cherries]] 
Bob     [type: human, foo: baz, colors: [agenta, blue, curquoise]]
Casper  [type: supernatural]

// We could draw edges based on attributes or do it on a node-by-node basis.
Bob -> [type: human]
[foods: apples] -> [foo: baz]
Bob -> Casper
\end{verbatim}
We would also be able to print out any node and see its attributes. Or print out an edge and see the attributes that activated it.

There are a lot of issues that we could choose to solve with this project. we could color or position the generated nodes in a way that might better represent the attributes in the data. For example, if a lot of nodes with certain attributes happen to be connected together, we would want to try and recognize that by coloring them all similarly.

\subsection*{Use Cases}
Here are some possible interesting usecases:

-Representing and maintaining someone's social network of friends as a graph, and changing this graph simply by altering the attributes of particular people. This data could be pulled automatically from someone’s profile to update the graph in real time.

Taking a list of people and grouping them based on personality trait attributes. Each personality trait would be a node, and the strength of the edges between nodes would reflect how many people had both traits.



\subsection*{Backend}

For the backend, we would like build on top of Graphviz in terms of displaying and manipulating graphs. 
For a programming language, our current plan is to build our project in Python. 


\end{document}
